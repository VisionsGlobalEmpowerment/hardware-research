%!TEX encoding = UTF-8 Unicode
\part{Project state}

\chapter{Overview}

There were two devices used for testing and investigation:
\textbf{Lenovo TB-7304i} and \textbf{Alldocube T1006S}. The devices
tested had notable differences in Android OS configurations so it
will be hard to maintain the same level of functionality across
different hardware models.

\chapter{Server-side}

Devices may be configured via \textbf{ADB over Wi-Fi} after some initial
setup and that functionality may be useful for configuration deployment
on multiple machines, but this case needs more investigation.

This functionality may be used not only for initial deployment but also
for massive device reset/rollback without interaction with users.

\chapter{Client-side}

Generally, standard parental control function is enough to provide the
necessary security level for children. Currently the configuration
process is manual and there is no way provided to ensure that multiple
devices will be configured alike.

Unfortunately \textbf{Alldocube T1006S} model provided had
\textbf{"Restricted profile"} functionality turned off making it
unsuitable for the task in device's unmodified state.


\chapter{Risks}

\begin{itemize}
\item \textbf{Current model with supposed software and/or user reset
done by a person won't work} because each setup takes lots of time and
is error-prone. This will lead to numerous setup errors and the project
will have all chances to stop because of this.
\item \textbf{There is a need for a single device model} because different
models from different vendors differs too much. This will lead to
growth of instructions, workarounds, tests and incur numerous problems
for product developers.
\item \textbf{Tests on Lenovo TB-7304i shown that Google Chrome is
inaccessible for Restricted Profiles} so the developers will have to
test the product with Android's \textbf{"Safe Browser"}. The problem
may be solved by lower-level Android OS configuration but this case
needs more investigation.
\end{itemize}

