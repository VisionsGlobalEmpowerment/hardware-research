%!TEX encoding = UTF-8 Unicode
\part{Project details}

\chapter{Server side}

It's advised to have as much functionality on server-side as possible
to reduce problems with client devices.


\section{Authentication}

There is software for WiFi HotSpots fencing called \textbf{CoovaChilli}:
http://coova.github.io/CoovaChilli/ . It's abandoned somehow but it's
the most functional solution for organizing fenced connection of
multiple users.


\section{Authorization}

Tasks like filtering the undesired traffic are also done better from
the server side - it'll be possible to avoid such problems like
adjusting tablets for this kind of work.


\section{Configuration deployment}

It is possilbe to create multiple user profiles from \textbf{ADB}
(Android Debug Bridge) but profile editing functionality needs more
investigation.


\chapter{End user side}

It's expected that the target audience is totally inexperienced with
computers so the less actions they will need to do to achieve the
desired result - the better.


\section{Devices}

There was investigation performed about which devices are popular and
may be used for the project. The devices that were considered "suitable
for project needs" are:

\begin{longtable}{|l|l|l|}
\hline
\textbf{Device Model} & \textbf{Device Characteristics} & \textbf{Comment} \\
\hline
Dexp Ursus S170i Kid's &
	\vbox{
		\hbox{\strut 7"}
		\hbox{\strut 1280x800}
		\hbox{\strut Android 7}
		\hbox{\strut 1 RAM/8 HDD}
		\hbox{\strut 3000 mAh}} &
	Preconfigured for parental control and has rubber frame \\
\hline
Huawei MediaPad T3 7.0 &
	\vbox{
		\hbox{\strut 7"}
		\hbox{\strut 1024x600}
		\hbox{\strut Android 6}
		\hbox{\strut 1 RAM/8 HDD}
		\hbox{\strut 3100 mAh}} & \\
\hline
Irbis TZ968 &
	\vbox{
		\hbox{\strut 9.6"}
		\hbox{\strut 1280x800}
		\hbox{\strut Android 7}
		\hbox{\strut 1 RAM/8 HDD}
		\hbox{\strut 4700 mAh}} & \\
\hline
Dexp Ursus P410 &
	\vbox{
		\hbox{\strut 10.1"}
		\hbox{\strut 1280x800}
		\hbox{\strut Android 8}
		\hbox{\strut 1 RAM/16 HDD}
		\hbox{\strut 5000 mAh}} & \\
\hline
\end{longtable}

Among multiple competitors the \textbf{Dexp Ursus S170i Kid's} model
considered interesting because of preconfiguration for parental control
and rubber frame.

Most of devices under \$ 100 have nearly same pararmeters so there is
nothing to choose from.


\section{Access limiting}

Android's standard profiles and parental control function is enough to
limit the end-user access to programs and whitelisted websites.

The account of the device owner may be protected by graphical key
making it rather problematic to crack.

It should also be noted that modern Android versions have device control
feature that may prevent the unauthorized user from device reset -
in case the device reset is performed without correct password - the
device may turn into brick and become nearly irreparable (may be
solved only by reflashing new OS on device).

The setup is point-and-click as for now so it won't work in case of
hundreds and thousands of devices but there is some ways (like using
Termux with configuration deployment tools) which must be investiated
further.


\subsection{Note on bypassing parental control function}

The problem boils down to graphical key bypass. There are
several ways to do it:

\begin{enumerate}
\item \textbf{Finding a graphical key}
The device inspected (Lenovo TB-7304i) had unlimited amount of
graphical key tries. The device makes 30-second pause each 10 tries.

It's possible to guess the key by inspecting traces left by fingers.
There are not very many variants of graphical key the user may remember
\item \textbf{Hard reset}
\item \textbf{Remote unblock using Google account}
It's possible to unblock the device remotely in case the Google account
is accessible.
\item \textbf{Nonstandard ways}
The user will have to preinstall special applications first.
\end{enumerate}


\subsection{Software for parental control}

There were multiple products (paid and free) tested but all products
have different limitations:

\begin{itemize}
\item \textbf{Kaspersky Safe Kids} - May be deinstalled by children;
\item \textbf{Norton Family} - Blocks websites only in "Safe Browser";
\item \textbf{Dr. Web Security Space} - Requires Internet access for
control purposes.
\end{itemize}

Other software is not functional enough to compete with programs
mentioned.

The conclusion is: Spending money on additional software does not
worth it for the project.


\section{Authentication}

Done transparently when device is connected to HotSpot via MAC. The
authentication and authorization is performed by CoovaChilli software
on server side.


\section{Rollbacks}

Android's new filesystem F2FS has snapshot and rollback functionality
but it must be investigated if it's used by existing devices.


\section{Threat model}

There is two kind of threats we expect:

\begin{enumerate}
\item Treats from external network: miners, viruses, etc.
\item Treats from users: curios children
\end{enumerate}

Here is the list of treats we're trying to mitigate:

\begin{itemize}
\item Getting full control on device by its end-user.
\item Getting sensitive personal information from device's
\textbf{"Emergency Information"}.
\end{itemize}

